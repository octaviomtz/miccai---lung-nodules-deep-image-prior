\documentclass{article}
\usepackage[utf8]{inputenc}

\title{miccai2019 - lung nodules priors}
\author{Octavio Martinez}
\date{March 2019}

\begin{document}

\maketitle

\section{Introduction}
Among all cancers, lung cancer is the one with the highest mortality rates (cite Siegel 2019). Lung cancer is usually diagnosed at later stages when treatment options are already limited. The most promising solution that leads to reduced lung cancer mortality seems to be screening with low-dose computed tomography, so that it may be detected early and while still in a treatable stage \cite{DeniseR2011}. \\

Lung cancer diagnosis is usually performed in two steps, with the first being pulmonary nodule detection and the second characterization of the identified nodules. Therefore, the current diagnostic procedure relies on nodules having appeared before lung cancer can be diagnosed. With this approach, however, only the area around the nodule is examined while other areas of the lungs that could potentially have diagnostic information of value are not studied.  

\end{document}
